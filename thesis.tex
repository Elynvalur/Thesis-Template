\documentclass[a4paper, hidelinks, dvipsnames, 11pt, listof=entryprefix, listof=nochaptergap]{scrreprt}
\usepackage[a4paper, left=32mm, right=19mm, top=19mm, bottom=19mm]{geometry}
\usepackage[utf8]{inputenc}

\usepackage[backend=biber, language=ngerman, style=authoryear-comp]{biblatex}
\addbibresource{references.bib}

%%% Allgemein
\usepackage[T1]{fontenc}
\usepackage{lmodern}
\usepackage[ngerman]{babel}
\usepackage{blindtext}
\usepackage{amsmath}
\usepackage{amssymb}
\usepackage{multicol}
\usepackage{utopia}
\usepackage[linktoc=true]{hyperref}

% When using babel or polyglossia with biblatex, loading csquotes is recommended to ensure that quoted texts are typeset according to the rules of your main language.
\usepackage{csquotes}

% Set correct breaks of urls in the bibliography
\setcounter{biburllcpenalty}{7000}
\setcounter{biburlucpenalty}{8000}

% Distanz zwischen der letztmöglichen Textzeile und des Seitennummer
\setlength{\footskip}{23pt}

% Einbinden externer PDF-Dokumente
\usepackage{pdfpages}

%%% Inhalt und Layout
% Einstellen des Zeilenabstandes
\linespread{1.25}

% The emptypage package prevents page numbers and headings from appearing on empty pages.
\usepackage{emptypage}

% Optimierter Randausgleich
\usepackage{microtype}

%%% Verzeichnisse
% Abkürzungen
\usepackage{acronym}

%%% Abbildungen
\usepackage{graphicx}
\usepackage{float}
\usepackage{wrapfig}

% Einstellen des fbox-Paddings
\fboxsep0mm

\begin{document}
    \interfootnotelinepenalty=10000
    \renewcommand{\bibname}{Literatur- und Quellenverzeichnis}
    \renewcaptionname{ngerman}{\figurename}{Abb.}
    \providecaptionname{ngerman}{\listoflofentryname}{Abb.}
    \providecaptionname{ngerman}{\listoflotentryname}{Tab.}

    % Titlepage
    \begin{titlepage}
    % Keine Darstellung von Seitenzahlen
  \thispagestyle{empty}
  
  \newgeometry{left=3.75cm, right=3.75cm, top=2cm}
    \begin{center}
  
        \begin{figure}[t]
            \centering
            \includegraphics[width=5cm]{resources/img/logo_hda.png}
        \end{figure}
  
        \vspace{0.5cm}

        \begin{large}
          Hochschule Darmstadt\\
          Fachbereich EIT
  
          \vspace{0.7cm}
  
          Masterarbeit\\
          zur Erlangung des akademischen Grades eines\\
          \vspace{0.2cm}
          \textbf{Master of Science}
        \end{large}
  
        \vspace{0.75cm}
  
        \begin{large}
          \textbf{Lorem ipsum dolor sit amet, consectetur adipiscing elit, sed do eiusmod tempor incididunt ut labore et dolore magna aliqua.}\\
        \end{large}
  
        \vspace{0.25cm}
  
        \begin{large}
          Ut enim ad minim veniam, quis nostrud exercitation ullamco laboris nisi ut aliquip ex ea commodo consequat\\
        \end{large}
  
        \vspace{0.75cm}
  
      vorgelegt von:\\
      Max Mustermann, 012345\\
  
      \vspace{0.75cm}
  
      Referent:\\
      Prof. Dr. Max Mustermann\\
  
      \vspace{0.25cm}
  
      Korreferent:\\
      Dr. Max Mustermann\\
  
      \vspace{0.75cm}
      erstellt bei der\\
      \textbf{Musterfirma SE}\\
      Abteilung
  
      \vfill
  
      
      Bearbeitungszeitraum:\\
      14.01.2022 - 25.03.2022
        
    \end{center}
\end{titlepage}
\restoregeometry

    \newpage

    % Genderpassus
    % Keine Darstellung von Seitenzahlen
\thispagestyle{empty}

\newgeometry{left=4cm, right=4cm}

\vspace*{\fill}

\begin{center}
  \begin{Large}
    \textbf{Sperrvermerk}
  \end{Large}
\end{center}

\vspace{0.5cm}

\noindent % Einrückung vermeiden
\glqq Die vorliegende Abschlussarbeit enthält unternehmensinterne Daten der Musterfirma SE. Daher ist sie nur zur Vorlage bei der Hochschule Darmstadt sowie den Begutachtern der Arbeit bestimmt. Für die Öffentlichkeit und dritte Personen darf sie nicht zugänglich sein.\grqq{}

\vspace*{\fill}
 
\restoregeometry
    \newpage

    % Genderpassus
    % Keine Darstellung von Seitenzahlen
\thispagestyle{empty}

\newgeometry{left=4cm, right=4cm}

\vspace*{\fill}

\begin{center}
  \begin{Large}
    \textbf{Hinweis: Gendergerechte Formulierung}
  \end{Large}
\end{center}

\vspace{0.5cm}

\noindent % Einrückung vermeiden
\glqq In dieser Arbeit wird aus Gründen der besseren Lesbarkeit das generische Maskulinum verwendet. Weibliche und anderweitige Geschlechteridentitäten werden dabei ausdrücklich mitgemeint, soweit es für die Aussage erforderlich ist.\grqq{}

\vspace*{\fill}
 
\restoregeometry
    \newpage

    % Inhaltsverzeichnis
    % Für einen Seitenumbruch im Inhaltsverzeichnis einfach vor die entsprechende \section ein   "\addtocontents{toc}{\protect\newpage}" stellen
    %\renewcommand\contentsname{Inhaltsverzeichnis}
    \tableofcontents
    \newpage

    % Abbildungsverzeichnis
    % \listoffigures
    % \newpage

    % Tabellenverzeichnis
    % \listoftables
    % \newpage

    % Abkürzungsverzeichnis
    \begin{multicols}{2}
    \include{acronyms.tex}
    \end{multicols}
    \newpage

    %%%%%%%%%%%%%%%%%%%%%%%%%%%%
    % Content

    % Hierachy:
    % Chapter
    %    Section
    %       Subsection
    %          Subsubsection

    \chapter*{Abstract}
    \addcontentsline{toc}{chapter}{Abstract}  
    \input{content/Abstract.tex}

    \chapter{Chapter 1}
\blindtext[1]

\section{The first Section}
\blindtext[2]

\section{The second Section}
\blindtext[2]

\blindtext[1]

\subsection{A Subsection}

\subsubsection{A Subsubsection}
\blindtext[1]

    \chapter{Chapter 2}
\blindtext[1]

    % Keine Darstellung von Seitenzahlen
\thispagestyle{empty}

\newgeometry{left=4cm, right=4cm}

\vspace*{\fill}

\begin{center}
  \begin{Large}
    \textbf{Ehrenwörtliche Erklärung}
  \end{Large}
\end{center}

\vspace{0.5cm}

\noindent % Einrückung vermeiden
\glqq Hiermit erkläre ich, dass ich die vorliegende Arbeit selbständig erstellt und keine anderen als die angegebenen Hilfsmittel verwendet habe. Soweit ich auf fremde Materialien, Texte und Gedankengänge zurückgegriffen habe, enthalten meine Ausführungen vollständige und eindeutige Verweise auf die Urheber und Quellen. Alle weiteren Inhalte der vorgelegten Arbeit stam- men von mir im urheberrechtlichen Sinn, soweit keine Verweise und Zitate erfolgen.

Diese Arbeit ist in gleicher oder ähnlicher Form noch keiner anderen Prüfungsbehörde vorgelegt worden.

Mir ist bekannt, dass ein Täuschungsversuch vorliegt, wenn die vorstehen- de Erklärung sich als unrichtig erweist.\grqq{}

\vspace{1.5cm}

{\raggedleft
Darmstadt, \today\\
}

\vspace*{\fill}
 
\restoregeometry

\end{document}